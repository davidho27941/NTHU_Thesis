\begin{abstract}  % Abstract 
The top quarks produced by proton-proton collisions in the Large Hadron Collider (LHC) undergo a very complicated process of formation. To date, the decay process of top quarks has not yet been well-classified, because of its complicated topology and large background events. In this project, we present a novel approach to the ``all hadronic decay'' process of top quarks based on the neural networks with attention mechanism, we refer to as the ``Symmetry Preserving Attention Networks'' (SPA-Net). These networks identify the decay products of each quark unambiguously and without combinatorial explosion(i.e.~the rapid growth of the complexity). This approach performs outstandingly compared to the generally accepted methods. Our networks can correctly assign all hadronic decay in 93.0\% of 6 jets, 87.8\% of 7 jets, and 82.6\% of $ \geq 8$ jets event respectively. The outstanding performance of this structure point the way to a more efficient solution of parton-jet assignment problem. A physical process contains a permutation symmetry in final states is a well-suited item to study with SPA-NET.
\newpage  % Independent page
\thispagestyle{empty}
\begin{center}
\vspace*{1.2in}
\large 摘要\\
\end{center}
\normalsize
在大型強子對撞機(LHC)實驗中,經由質子對撞所產生的頂夸克對具有非常複雜的過程以及產物,至今仍無法被非常正確的判別以及重建。在本研究中,我們提出了一個利用新穎的機器學習方法來對雙頂夸克全強子衰變過程進行重建。此方法基於Attention mechanism,我們稱之為Symmetry Preserving Attention Networks(SPA-Net)。這個模型架構可以在避免組合性爆炸的前提下對所有的衰變產物進行辨識以及重建。此方法對比於傳統的$\chi^{2}$重建方式,表現出了非常巨大的差異。本方法可以在一、 存在6 jets條件下正確的重建93\%的事件;二、存在7 jets條件下正確的重建87\%的事件;三、存在大於8 jets條件下正確的重建82.6\%的事件。此架構的出色表現指明其具有更大的潛能能以更有效的手段進行事件重建。對於本架構而言,只要是一個在最終產物具有置換對稱性的物理過程,就可以是一個適合且有潛力的套用對象。

\end{abstract}



