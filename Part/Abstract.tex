\begin{abstract}  % Abstract 
The top quarks produces by $pp$ collision in Large Hadron Collider(LHC), have a very complicated process still can't be well-classified today. In this project, we present a novel approach to the ``all hadronic decay'' process of Top quarks base on the neural networks with attention mechanism, we call it ``Symmetry Preserving Attention Networks''(SPA-Net). This networks  identify the decay products of each quarks unambiguously and without combinatorial explosion. This approach perform a outstanding result compare to the existing state-of-the-are method. Our network can correctly assigning all hadronic decay in 93.0\% of 6 jets, 87.8\% of 7 jets, and 82.6\% of $ \geq 8$ jets event respectively.
\newpage  % Independent page
\thispagestyle{empty}
\begin{center}
\vspace*{1.2in}
\large 摘要\\
\end{center}
\normalsize
在大型強子對撞機(LHC)實驗中,經由質子對撞所產生的頂夸克對具有非常複雜的過程以及產物,至今仍無法被非常正確的判別以及重建。在本研究中,我們提出了一個利用新穎的機器學習方法來對雙頂夸克全強子衰變過程進行重建。此方法基於Attention mechanism,我們稱之為Symmetry Preserving Attention Networks(SPA-Net)。這個模型架構可以在避免組合性爆炸的前提下對所有的衰變產物進行辨識以及重建。此方法對比於傳統的$\chi^{2}$重建方式,表現出了非常巨大的差異。本方法可以在一、 存在6 jets條件下正確的重建93\%的事件;二、存在7 jets條件下正確的重建87\%的事件;三、存在大於8 jets條件下正確的重建82.6\%的事件。

\end{abstract}



