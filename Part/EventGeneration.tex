\chapter{Event Generation}\label{Event Generation}

%This chapter is organized as follow. Section \ref{VBF:CommonSelection} provides ...


\section{Simulation methods}\label{sec:MC sample}
The procedure of preparing dataset is very important in the ML workflow. A set of Monte Carlo simulator had been used to generate our dataset, including MadGraph\_aMC@NLO (v2.7.2), Pythia8 (v.8.2), and Delphes (v3.4.2). Those simulation helped to calculate showering, hadronization, and detector simulation. We applied the ATLAS parametrization during detector simulation.  The data are generated at Leading Order (LO) quantum chromodynamics (QCD) and using the PDF set NNPDF23\_lo\_as\_0130\_qed. The top mass is configured as $m_{\rm top}=173$ GeV. The W quark decay is forced hadronically into a $(q, q')$ pair. The following is our configuration:
\\
\begin{adjustbox}{max width=\textwidth}
\centering
\begin{lstlisting}[caption={Configuration for generating samples. The ``iseed'' is just a placeholder, it will be changed when generating samples.},captionpos=b]
	generate p p > t t~ QED=0, (t > W+ b, W+ > j j), (t~ > w- b~, w- > j j ) 
	output <file_path> 
	launch <file_path> 
	shower=Pythia8  
	detector=Delphes 
	analysis=OFF 
	done  
	set nevents = 10000 
	set iseed = 1 
	Delphes/cards/delphes_card_ATLAS.tcl
	done 
	exit 
\end{lstlisting}
\end{adjustbox}
\\
To get a more general performance, we scan the iseed value from 1 to 30000, each value has 10 thousand events before event selection. The reason for scanning iseed value is that the iseed value is the key to the random data generation. Originally, the program will choose the iseed value randomly and generate different samples. By scanning the issed value, we can make sure the seed is not reused. 





