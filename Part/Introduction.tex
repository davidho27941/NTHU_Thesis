\pagenumbering{arabic} %從這頁開始用數字頁碼

\chapter{Introduction} 
%cite放在./Part/Reference,如果Thesis.tex第一次看到這個位置,他不知道要到下面去找,要編譯兩次才會出現引用編號。

\newcommand{\comment}[1]{}
%==================== Higgs ====================%

Inside the Large Hadron Collider(LHC), two protons collide with very high energy and produce many kinds of products. A process whereby $pp$ collision produces a pair of top quarks and results in the 6 jets final state is called \textbf{All Hadronic Top-quark-pair Decay}. This process has a very complicated signature due to a large number of combinations producde by the possible permutation of final state jets. These jets produced by the top quark pair are hard to tag as a specific daughter of top quarks correctly. The traditional method is to reconstruct the event using $\chi^{2}$ reconstruction, but it takes a long time to compute and the result have a low accuracy (about 30\% or less). The investigation of top quark and its full hadronic decay channel is 1. Top quark is the most heaviest  fundamental particle in the standard model and will decay before hadronization, 2. The branching ratio of full hadronic decay is the biggest component of Top quark decay(46\%). 
\\
For a problem that contains a large amount of data and highly require complex and intensive computing resources, machine learning can widely provide powerful support on solving the problem and helps to reduce the CPU time. The machine learning method can facilitate the study and discovery of physics phenomena,an example of which is the remarkable discovery of the Higgs Boson. Both CMS and ATLAS groups apply machine learning methods to promote the search for the Higgs Boson. \cite{Aad:2012tfa}\cite{Chatrchyan:2012ufa}
\\ 
In this thesis, we developed a novel architecture for the parton-jet assignment problem. This method is base on the state-of-the-art machine learning technology, Attention mechanism.\cite{A.Vaswani:2017} We call this novel ML model \textbf{Symmetry Preserving Attention NETworks (SPA-NET)}. By applying attention networks, the SPA-NET is capable of outstanding performance compared to traditional methods while avoiding combinatorial explosion. And thanks to the natural properties of the attention network, the network reflects the permutation symmetry naturally and provides a chance to explore the model with permutation symmetry. 
\\
This project was accomplished in collaboration with distinguished physicists from the University of Washington(Shih-Chieh Hsu), University of California Irvine(Mike Fenton, Alexander Shmakov, Daiel Whiteson, and Pierre Baldi). Mike provided an idea of the suitable process to investigate.  My jobs was focused on the physics concept, designed the data format, and generate datasets. Also, the traditional event reconstruction method is implemented by my effort. Alexander provide a technical support and machine learning network setup. This project has been submit to the arxiv and under the review of PR. D.\footnote{There are two version of submission, \hyperlink{https://arxiv.org/pdf/2010.09206.pdf}{https://arxiv.org/pdf/2010.09206.pdf} and \hyperlink{https://arxiv.org/pdf/2106.03898.pdf}{https://arxiv.org/pdf/2106.03898.pdf}}\footnote{A full code repository containing a general library, the specific configuratio used, and a complete dataset release, are avaiable at \hyperlink{https://github.com/Alexanders101/SPANet}{https://github.com/Alexanders101/SPANet}}
\\
Top physics and the concept of machine learning in chapter 2; event generation and simulation configuration are explained in chapter 3. Dataset and event reconstruct using the traditional method and ML approach is explained in chapter 4. Result are shown and discussed in chapter 5; summary and conclusion are presented in chapter 6.


